\chapter{Conclusion}
\label{chp:conclusion}

In the last years, Trusted Execution Environments grew in importance due to 
their application in cloud infrastructures.
These technologies allow one to define \emph{trusted regions} that benefit from 
either \emph{memory isolation} and \emph{remote attestation}.
The former ensures the content of a trusted region is shield even against a 
fully compromised environment. 
The latter, instead, is a cryptographic protocol that enables a third party to 
communicate with a trusted region correctly bootstrapped. Regardless of the new 
security achievements of these technologies, TEEs still suffer from scalability 
issues. 
Moreover, they alone cover only a subset of the attack vectors that could 
appear in the wild.

In this thesis, we investigated the limitations that affect \emph{memory 
isolation} and \emph{remote attestation}, studied possible new threats, and 
proposed novel solutions to improve their security guarantees. For what 
concerns \emph{memory isolation}, we discussed its limitations concerning the 
restricted size of the protected area.
Then, we investigated new intrusion techniques that exploit a 
\emph{trusted region}, and their implication in case of incident response. 
In terms of \emph{remote attestation}, instead, we studied new models that 
capture runtime properties, overcome scalability issues, and propose new 
protocols that can protect advanced TEE modules.


In Chapter~\ref{chp:static-protection}, we proposed a software design that 
overcomes memory isolation limitations. It combines novel anti-tampering 
techniques that leverage trusted computing technologies.
We achieved this by adopting a packing strategy that is similar to the one used 
by malware to hide its functionality. 
Our approach forces a program to call trusted functions to be executed properly 
(by unpacking a piece of software from a trusted container). 
We implemented a proof-of-concept prototype of our technique by using Intel 
Software Guard eXtension (SGX) technology.
We illustrated our approach by protecting an agent that was designed to collect 
user’s events and ship them to a central server. 
Through this implementation, we showed how our architecture can guarantee 
further security properties such as a secure installation and continuous client 
monitoring. Using our prototype, we measured the overhead in terms of
lines of code (less than $10$ lines), execution time (on average $5.7\%$ more), 
and space required for the trusted container ($300KB$).
To sum up, our approach results in a scalable and software protection solution 
that overcomes TEE intrinsic limitations.


In Chapter~\ref{chp:advanced-threats}, we explored new threats that take 
advantage of the \emph{memory isolation} from modern TEEs.
Specifically, we proposed a new stealthy code-reuse attack that minimizes its 
presence against a healthy OS.
Our intuition is to implant a backdoor inside the victim \emph{trusted region}. 
Consequently, an adversary just needs a minimal trigger without repeating the 
attack from scratch.
We implemented our idea in SnakeGX, which is a framework to install backdoors 
in SGX enclaves that behave like additional secure functions.
SnakeGX extends and adapts to the strict SGX environment the concepts of 
data-only malware \citep{vogl2014persistent}.
In particular, SnakeGX has a reliable context-switch mechanism based on a newly 
discovered design error of the Intel Software Development Kit for SGX, which we 
reported to Intel.
We evaluated our findings against StealthDB, an open-source project that 
implements an encrypted database.
Our experiments show that we can reduce the memory footprint of the payload 
while preserving the enclave functionality.
Finally, we released an open-source version of our proof-of-concept for the 
community.

In Chapter~\ref{chp:forensic}, we presented a set of practical guidelines to 
inspect the physical memory acquired from an SGX-enabled machine.
Our work covers many aspects of memory forensics, such as retrieving EPC zones 
and \emph{enclaves} information, rebuilding kernel structures, and analyzing 
the \emph{enclave} interfaces and layout. This information can help 
human analysts to identify the presence of enclaves, their corresponding
user-space processes, and kick-start manual reverse engineering 
by pointing out the functions responsible for the communication to, and from,
the enclave code. 
We implemented these techniques in an acquisition tool and a set of
Volatility plugins and show how they can be applied to the analysis of $45$
samples and three real use-case scenarios.

In Chapter~\ref{chp:runtime-protection-untrusted}, we proposed ScaRR, the first 
schema that enables runtime RA for complex systems to detect control-flow 
attacks generated in user-space.
ScaRR relies on a novel control-flow model that allows to:
\begin{enumerate*}[label=(\roman*)]
	\item apply runtime RA on any software regardless of its complexity,
	\item have intermediate verification of the monitored program, and
	\item obtain a more fine-grained report of an incoming attack.
\end{enumerate*}
We developed ScaRR and evaluated its performance against the set of tools of 
the SPEC CPU 2017 suite.
As a result, ScaRR outperforms existing solutions for runtime RA on complex 
systems in terms of attestation and verification speed, while guaranteeing
limited network traffic.
Future works include: investigating techniques to extract more precise CFG, 
facing compromised operating systems, and studying new verification methods for 
partial reports.

In Chapter~\ref{chp:runtime-protection-trusted}, we proposed SgxMonitor, a 
novel remote attestation anomaly detection schema for SGX enclaves. As enclaves 
are designed to secure code that performs specific security- and 
privacy-sensitive tasks, SgxMonitor relies on a combination of symbolic 
execution and static analysis to model the expected behavior of enclaves with 
high code coverage and low false positives.  
We assessed SgxMonitor across four real use cases (\ie \textsf{Contact}, 
\textsf{StealthDB}, \textsf{libdvdcss}, \textsf{SGX-Biniax2}) and a 
\textsf{unit test} to validate enclave's corner cases. SgxMonitor overhead is 
comparable to the state-of-the-art of remote attestations (a median of $260$K
\emph{action}/s), whereas its macro-benchmark overhead and high precision
with $96$\% code coverage and zero false positives support SgxMonitor in 
realistic deployments to detect anomalous runtime executions of SGX enclaves.

\vspace{0.3cm}
Following the thesis direction, academia and industry should invest more 
effort for deeply understanding the implications of TEE in real scenarios.
For the market to really accept these technologies as a new standard, vendors 
should study solutions that overcome the TEE intrinsic scalability issues. 
In Chapter~\ref{chp:static-protection}, our new design is only a first step for 
smarter software architecture for TEE.
Much work is left to be done in other TEE technologies, such as ARM TrustZone 
\citep{arm-trustzone} or Apple Platform Security \citep{apple-enclave}, more 
common in smartphones.
As a consequence, having more isolated \emph{trusted regions} will lead to new 
threats, as highlighted in Chapter~\ref{chp:advanced-threats} by SnakeGX.
However, SnakeGX addresses only one part of the problem. 
New completely autonomous data-only malware, which does not require an external 
trigger, could appear shortly (similar to the work of 
\cite{vogl2014persistent}).
Moreover, data-only malware could easily appear in other TEE technologies, such 
as ARM TrustZone. In particular, this latter provides an enriched OS entirely 
contained in the isolated memory 
regions, namely \emph{Secure OS}~\citep{ngabonziza_trustzone_2016}.
In a nutshell, the Secure OS works as a bridge between the \emph{trusted 
applications} (inside the Secure OS) and the Normal OS (outside the secure 
memory regions); resembling the interaction between an SGX enclave and the host 
application (see Section~\ref{ssec:development-frameworks}).
The internal management of the Secure OS requires additional complexity. 
Ideally, this allows more complicated applications to run in ARM TrustZone. 
However, this also implies a larger code-base and many unused memory regions 
that may assist a malware developer to design more sophisticated code-reuse 
based threats (like SnakeGX).
When this happens, the security community will be unprepared to dissect such 
threats, as discussed in Chapter~\ref{chp:forensic}, thus arising the necessity 
of more advanced malware dissection techniques that
can penetrate the TEE boundary. A promising direction is represented by 
offensive forensic techniques that could use micro 
architectures drawbacks as pinlocks for \emph{trusted regions} 
\citep{offensiveforensic}.

Having the capability to bypass the TEE \emph{memory isolation}, and 
consequently take control of trusted  \emph{memory regions}, arises the need 
of new defenses beyond the current \emph{remote attestations} schemes.
In this scenario, the models in chapters~\ref{chp:runtime-protection-untrusted} 
and~\ref{chp:runtime-protection-trusted} just scratch the surface of 
possible solutions as they merely indicate directions and new challenges for 
more comprehensive approaches.
Besides classic ROP techniques, data-only attacks can be used to bend the 
enclave execution without violating the code integrity, and the community 
should be ready to face these threats.
For tackling these attacks, runtime RAs should include data-flow information.
First examples have been explored, but they mainly fit embedding systems 
\citep{sun2020oat,aberadiat}.
More importantly, current runtime RAs introduce an important overhead 
that makes them impractical in real scenarios.
From here the need of tracing \emph{trusted regions} at hardware level.
Besides few research prototypes for embedded systems 
\citep{Dessouky:2018:LLH:3240765.3240821}, also major CPU vendors are proposing 
hardware-level protections, such as Intel CET \citep{intelcet} and ARM 
\citep{armpa}.
However, these solutions require further development since the academic 
community already managed to bypass these controls \citep{van2012memory}.
Finally, an advanced research challenge could be combining 
automatic software partition methods (\eg Glamdring -- 
\cite{lind2017glamdring}) with runtime RA (\eg SgxMonitor).
In short, Glamdring automatically splits an application into secure and 
unsecure components by the means of taint analysis. In this way, a developer 
indicates small critical pieces of code, and Glambdring understands the main 
dependences and wraps them into protect \emph{memory regions}, also including 
additional checks against trivial attacks. 
We can combine works like Glamdring with SgxMonitor. The new runtime RA schema 
could include an enriched model that takes into consideration the state of 
multiple secure containers altogether. Adopting this protection will require an 
adversary to mimicry a more complex model that is the combination of many 
enclaves, thus forcing multiple parallel attacks.

In conclusion, this thesis discusses five contributions in the field of TEEs to 
overcome current limitations, explore new threats, and propose mitigation.
The works in chapters~\ref{chp:static-protection}, \ref{chp:advanced-threats}, 
and~\ref{chp:forensic} dig into \emph{memory isolation} while 
chapters~\ref{chp:runtime-protection-untrusted} 
and~\ref{chp:runtime-protection-trusted} propose new models that extend 
\emph{remote attestations} schemes.
We hope the results achieved by these works will help TEE designers and 
security experts to better understand the role of these new technologies in 
an even more connected world.


